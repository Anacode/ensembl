\documentclass[11pt,a4paper]{article}

\usepackage{
  fancyheadings,lscape,
  rotating,float,longtable,epsfig
}

\newcommand{\sdp}{\setlength{\baselineskip}{16truept}} %double spacing

\begin{document}

\title{Ensembl. Open Genome Annotation}
\author{ensembl-dev@ebi.ac.uk\\www.ensembl.org\\Last time seriously looked at - May 2001}


\maketitle

\newpage
\tableofcontents
\newpage

 
\newpage

\section{In 10 lines or less...}

Ensembl is a software system which supports automatic annotation of
genome projects, optimised for large, eukaryotic genomes. The best
example of this is the annotation of the human genome, which Ensembl
is one the leading groups. In particular Ensembl can annotate rapidly
updating fragmentary genomes with large introns. For most other systems
this is a challenge.

Ensembl is an entirely open software project: you can use its data
without restriction and the code with very few restrictions. This
document is aimed at getting bioinformatics developers over the
learning curve in understanding Ensembl.

\section{Before you start}

This document is going to assumme you understand the basics of
bioinformatics; if the words ``BLAST'', ``STS Marker'' or ``reverse
strand'' confuse you then the rest of this document probably will be
indecipherable. As well as being a reasonably complex (but only where
it has to be), Ensembl is a large software project. The code for
example is split up across 9 and counting cvs modules and has been
developed over 2 years (in the case of bioperl, more like 6 years)
with currently 18 active developers on the project. There are more
than 100 different objects in Ensembl that get combined in different
ways.  It is also worth visiting the Bioperl project,
http://bio.perl.org (which Ensembl uses as a base library) and read
tutorial documentation there. All in all this is not a project you
will get to know within one or two hours of playing around. That said,
the next section - first steps with Ensembl - will get you over the
first speed bump. However before we jump into the first steps, a quick
answer of some often asked questions:

\begin{description}

\item[Why Perl? Why not Python or Java?] We have a love-hate
relationship with Perl, but the basic answer is threefold: (a) we knew
it would work as we had experience in writing systems that scaled to
this level in Perl (b) We had Bioperl as a basic starting point
(Biojava was not mature when we started and is still someway from what
we need) (c) we knew we could hire people with Perl and molecular
biology knowledge.

\item[Why MySQL? Why not Oracle? or AceDB?] We have been very happy
with MySQL.  For us it was an easy RDB to learn (light on the DBA
aspects) and has scaled well to very large loads and tables. We have
not found the lack of transactions or views showstoppers for our work
- so far. With AceDB we did not have confidence that it would scale to
the sort of problems we expected to have.

\item[Does it really have to be this complex?] Mainly yes. There are a
number of places where we don't help ourselves because the code is old
and in light of a better understanding, sometimes somewhat off the
mark. This is what you expect in large projects

\item[Are you sure? I have run BLAST and Genscan myself...] Honestly,
if we can make it simpler, we will! Running BLAST and Genscan yourself
and doing small stuff is pretty easy. Alot of what Ensembl does at the
basic end is this easy, but there are one or two areas where things
get much more complex and the scale of the problem makes even doing
simple things hard.

\end{description}

Enough waffle! Let's do something useful

\section{Using Ensembl as a datasource}

You can treat Ensembl mainly as a source of well structured genomic data, and not
get involved with the software that builds it nor the software which allows easy 
biological interpretation. Main people find this data-only access one of the most
useful things Ensembl offers.

There is a mysql instance which we keep open to the internet (outside
the firewall), called kaka.sanger.ac.uk.  (It is unclear whether the
association of kaka - mean ``shit'' in many countries was deliberate
or just a happy coincidence. It does give you an idea of the quality
of this machine). If you have the myql client - available on all modern
linux distributions - if you don't have it go to www.mysql.com you can open
a terminal to it, username anonymous

\begin{verbatim}
birney@monkey> mysql -u anonymous -h kaka.sanger.ac.uk
\end{verbatim}



The most current database release is called ``current''. Using the
mysql command ``show databases'' should give you all the databases to
look at.


if you start running large queries or downloads, you may well want to
bring the MySQL instance local. MySQL files are found on the ftp site
which can be uploaded by following the instructions there (basically
use all.sql to initialise the database followed by mysqlimport)

To understand how to probe the data you need to have a working
knowledge of SQL, a working knowledge of basic genomic data types and
also our schema. Sadly we do not (yet) keep a nice entity relationship
style diagram in sync with our schema and also the schema has
accummulated a certain amount of cruft over time. To see our current
schema, view the DDL statements in ensembl/sql/table.sql (go via
WebCVS or anonymous CVS) or use the ``show tables'' and ``describe
XXXX'' syntax in the mysql client. The core part of the schema has
basically remained constant over the last year and is as follows:

\subsection{Clone, Contig and DNA tables}

These tables hold the underlying DNA sequences and container accession numbers.

\begin{description}
\item[clone] The Clone table represents one entry from the
EMBL/GenBank databanks (ie, one accession number). The id column is
the accession number. The internal\_id column is a unique id which 
other tables refer to
\item[contig] The Contig table represents a single assembled (usually by
Phrap) piece of DNA. For finished clones there is one contig per clone. For
unfinished (HTGS) clones there are multiple contigs per clone - in the EMBL/GenBank
entry these are usually separated by a ``standard'' number of Ns. The id column
is an entirely Ensembl generated text id for this contig. The internal\_id is a
unique integer referred to by other tables. The clone column is an integer referring
back to the clone column. The offset column indicates where in the EMBL/GenBank
coordinate system this contig starts. Many other tables link to contig via its
internal\_id. Finally in software the information here is called ``RawContig'' to 
differentiate itself from the the in-software assembled ``VirtualContig''
\item[dna] The DNA table represents the DNA sequence for the contig. There is a
one to one relationship with contig - the separation is because one can imagine updating
contig information without updating the DNA table (though we never do this) and 
one can imagine in the future running a more space efficient or across-network storage
of DNA.
\end{description}

\subsection{Feature,Fset,Fset\_Feature,RepeatFeature,Analysis tables}

These table hold the ``Raw Computes'' which are the assembly insensitive computes,
such as GenScan, BLAST and RepeatMasker. 

There were some bad design decisions here made about 2 years ago which
we have yet to unpick. Firstly the Feature table and the RepeatFeature
table should be merged.  Secondly I think we would prefer to drop the
Fset/Fset\_Feature (indeed very little active code uses these tables)
and move Genscan computes across into the Gene tables (detailed next).

\begin{description}
\item[Feature] This is the main table for the computes. The two main tables this
links out to is contig via the contig.interal\_id and analysis via the analysis
link. Analysis has one row per compute type and its columns are pretty self
explanatory. The position in the contig coorindates of this feature is given
by seq\_start,seq\_end and strand. Where sensible the position in the sequence
it is being compared to is in hstart,hend,hstrand with hid being the identifier
(we try hard to use accession numbers in this field, but do not always succeed).
\item[RepeatFeature] just like above but only for repeats. hid is repeatmasker repeat
type
\item[Fset,Fset\_Feature] This was meant to model GenScan features,
which were treated as an ordered list of features. However this
rapidly became impractical as the join from fset through fset\_feature
took too long. As the Genscan order was also implicitly in the
ordering of the features in the DNA, it was deduced by that (take a look
at get\_all\_PredictionFeatures in RawContig.pm to get an idea of this). 
\end{description}

\subsection{Gene,Transcript,Exon tables}

The Gene,Transcript,Exon tables hold information about predicted genes. When the
Ensembl analysis pipeline is running, these tables hold the intermeadaite 
GeneWise and other gene build results. On a finished datasource however only
the final ``GeneBuilder'' genes are present.

Again a bad design decision was made (which we are now regretting) not have true
internal ids here. The varchar ids are our own assigned but external visible ids.

\begin{description}
\item[exon] Exons are the only gene centric objects which connect to
sequence.  They do this via the contig column and
seq\_start,seq\_end,strand. The sticky\_rank is for single exons which
cross rawcontig boundaries - we should explain this better somewhere 
I guess.
\item[transcript] Transcripts have an ordered list of exons: however, one
exon can be in more than one Transcript. The Exon\_Transcript table holds
the ordered join from transcript to exon using the rank column. Transcripts
have a translation object which is how start and end translation points
are held.
\item[gene] Genes have an unordered set of Transcripts. This is managed by
the gene column in Transcript
\item[translation] The Translation table holds the start codon exon and
position in that exon of the start codon, and similarly for the stop codon.
The positions are relative to the exon, not in any absolute or contig relative
coordinate system (this is important for sticky exons)
\end{description}

\subsection{Assembly information - StaticGoldenPath table}

The static\_golden\_path table holds the assembly information. Ensembl
can store (but rarely does) multiple assemblies at the same time. The
assembly information is stored as a way to fit together raw contigs
into the final assembly. This is stored a list of segments in the
table, with raw\_id pointing to contig table and raw\_start, raw\_end
and raw\_ori indicating which part and which orientation that raw part
should fit.

The other columns deal with two other assembly coordinate systems -
chromosome and fpc contig. FPC contig has now become an entirely
artifical intermedaite coordinate system with a number of contigs
forming a chromosome. There is alot of redundancy amongst the
different coordinate systems here - for example, fpc contigs can be
described as offsets into chromosomes.

You may have noticed that nothing inside Ensembl is stored in
chromosomal coordinates directly, but always in RawContig coordinates.
To map things to chromosomal coordinates you need to have SQL like:

\begin{verbatim}
  SELECT rf.id,
  IF     (sgp.raw_ori=1,(rf.seq_start+sgp.chr_start-sgp.raw_start-$glob_start),
         (sgp.chr_start+sgp.raw_end-rf.seq_end-$glob_start)) as start,                                        
  IF     (sgp.raw_ori=1,(rf.seq_end+sgp.chr_start-sgp.raw_start-$glob_start),
         (sgp.chr_start+sgp.raw_end-rf.seq_start-$glob_start)), 
  IF     (sgp.raw_ori=1,rf.strand,(-rf.strand)),                         
         rf.score,rf.analysis,rf.hstart,rf.hend,rf.hid  
  FROM   repeat_feature rf,static_golden_path sgp
  WHERE  sgp.raw_id = rf.contig
  AND    sgp.chr_end >= $glob_start 
  AND    sgp.chr_start <=$glob_end
  AND    sgp.chr_name='$chr_name' 
  ORDER  by start";
\end{verbatim}

We are investigating easy denormalisation steps and hope to offer denormalised
query ability in the future.

\subsection{Other tables}

There are a number of other tables. Some are useful - others are vestigal tables
which we haven't decrufted. We should decruft tables no doubt!


\section{Installing Ensembl Software}

Using Ensembl as a data source is one thing, but pretty soon you will
want to join the data up into sensible results - things like protein
translation or masked DNA sequence or some other manipulation. Rather
than writing this software yourself you can use ours which we have
built up over the years.  Nearly everything inside Ensembl uses this
software layer (if you have a corporate IT background, you can think
of this as our business objects).


To use the Ensembl software you need to download and install the
following packages (the order listed is the best order to download
them in).

\begin{description}
\item[MySQL] Fetch from www.mysql.com. Unpack and do the standard
``./configure'', ``make'' (become root) ``make install''. Use --prefix
on the configure script to install somewhere other than the
``standard'' places. If you want to run your own local copy of Ensembl
you will need to initialise the databases and start the mysql
daemon. Start the mysql daemon by going ``mysql\_install\_db'' (you
only do this once to initalise the mysql installation) followed by
``safe\_mysqld'' to start the mysql daemon. Read more in the Mysql
docs to get an understand of what this means.
\item[DBI] DBI is the perl bindings to relational databases. The easiest way to install
this is using the in-build installation system in Perl, called
CPAN. Become root, go ``perl -MCPAN -e shell'', at the cpan prompt go ``install DBI''.
Alternative download the DBI package from CPAN and install by going ``perl Makefile.PL'',
``make'', (become root), ``make install''.
\item[MySQL DBI drivers] DBI only provides the framework to use a relational database:
you need the MySQL drivers. Again, using CPAN (``perl -MCPAN -e shell'') go ``install DBD::Mysql''.
\item[Bioperl] Ensembl sits ontop of Bioperl as a basic bioinformatics library (many of the
Ensembl developers also contribute to Bioperl). Unlike the previous
packages in which ``the latest version'' is likely to work, there is
more of a coupling between Ensembl and bioperl. Currently Ensembl
0.8.0 has a dependency on Bioperl 0.6.2 (in fact the head of the 0.6
stable branch for bioperl - but you would have to be using Ensembl in
mind numbing detail to notice the difference).  At the moment, most
but certainly not all, of Ensembl will work with the forthcoming 0.7
series of bioperl, and it will throw alot of warnings. To download, again you can
use cpan with ``install bioperl-0.6.2.tar.gz''. If you are reading this document past Ensembl0.8
you'll have to root around on our web pages to see which bioperl to use with the Ensembl you aim
to download.
\end{description}

Apologies for the number of external packages to install. This is I'm afraid to say 
par for the course for this sort of project.

You are now ready to download Ensembl. You have two basic options of
what to download.  (a) Download a tar file of the code which is
synchronised with an Ensembl data release.  You will find this in
ftp://ftp.ensembl.org/pub/current/software/ (b) Download the leading
edge version using anonymous cvs. The cvs repository is
cvs.sanger.ac.uk, username cvs, password CVSUSER (case important). You
want to check out the ensembl module

If anonymous cvs makes no sense to you, go for the tar ball.

Poke your nose into the ensembl directory. You should have a number of directories there, being
modules, scripts, docs and other stuff. You want to cd into modules. This is a top level 
perl modules directory. Go ``perl Makefile.PL'' followed by ``make test''. This will run the
Ensembl test system.

The Ensembl test systems fails for 3 main reasons.

\begin{description}
\item[No Bioperl] Phenotype: Could not find Bio::Root::RootI in @ISA. You have not installed Bioperl
(see above!). Alternatively make sure Bioperl is in your PERL5LIB.
\item[No DBI/MySQL DBD] Phenotype: Could not find DBI in @ISA. You have not installed DBI or the
MySQL DBD drivers. See above 

\item[No permissions on local MySQL]
Phenotype DBI connect failed: Unknown MySQL Server Host localhost not
found.  

\end{description} 

The first two is just your error. The third is more
understandable. Ensembl needs to connect to a local mysql to run tests
(it loads and dumps mini mysql databases for testing purposes). By
default it connects to localhost mysql as user root with no
password. This is a bad - but default - connection protocol!

To change this copy the file in modules/t called
EnsTestDB.conf.example to EnsTestDB.conf (also in modules/t), and edit
the configuration to match your system: in particular you want to
connect to a database as a user that has drop and create database
priviledges. The EnsTestDB.pm module will create a unique database
name for each test, so you need not worry about it ``trashing'' other
pieces of information.

When the test system runs, it will output alot of stuff to the terminal, but hopefully say
``All Tests Successful'' at the end. It is really worth getting the test system to run, as then you
will be confident that Ensembl actually works on your system as you embark on more complex processes.

\section{First script}

The first script we'll write is to dump the underlying DNA fragments
which make an assembly out as fasta format. The database to use is
the internet accessible kaka.sanger.ac.uk 

\begin{verbatim}

#!/usr/local/bin/perl

use Bio::EnsEMBL::DBLoader;
use Bio::SeqIO;


# this will die if if can't load this database
$db = Bio::EnsEMBL::DBLoader->new('Bio::EnsEMBL::DBSQL::DBAdaptor/host=kaka.sanger.ac.uk;user=anonymous;dbname=current');

# build seqio output stream

$seqout = Bio::SeqIO->new( '-format' => fasta, '-fh' => \*STDOUT);

# get out all the accession numbers (clone ids) for
# sequences in this database

@clones = $db->get_all_Clone_id();

my $index = 0;
# for each clone_id, get the clone
foreach $clone_id ( @clones ) {
	$clone = $db->get_Clone($clone_id);
	foreach $fragment ( $clone->get_all_Contigs() ) {
		$seqout->write_seq($fragment);
   	}
	$index++;
	if( $index > 10 ) {
		last; # to prevent the entire database being shipped across
	}
}

\end{verbatim}

To explain this in more detail:

\begin{verbatim}

#!/usr/local/bin/perl

use Bio::EnsEMBL::DBLoader;
use Bio::SeqIO;

\end{verbatim}

This loads up the necessary Perl modules. Only one Perl module for Ensembl is needed: DBLoader.
DBLoader will at run time load up the modules required for loading the database of choice

\begin{verbatim}
# this will die if if can't load this database
$db = Bio::EnsEMBL::DBLoader->new('Bio::EnsEMBL::DBSQL::DBAdaptor/host=kaka.sanger.ac.uk;user=anonymous;dbname=current');
\end{verbatim}

This connects to the actual database. The string used for DBLoader is
a sort of combined software+location string. The first part indicates
that the Bio::EnsEMBL::DBSQL::Obj is the software module which will
negotiate the Ensembl objects from the database. The remainder are
parameters used in that modules construction, in this case the host,
user and dbname of the database

\begin{verbatim}
$seqout = Bio::SeqIO->new( '-format' => fasta, '-fh' => \*STDOUT);
\end{verbatim}

This builds a standard Bioperl output stream. In this case to Fasta format to stdout.

\begin{verbatim}
@clones = $db->get_all_Clone_id();


# for each clone_id, get the clone
foreach $clone_id ( @clones ) {
	$clone = $db->get_Clone($clone_id);
	foreach $fragment ( $clone->get_all_Contigs() ) {
		$seqout->write_seq($fragment);
   	}
}
\end{verbatim}

This is the real guts of the system. For each id, it is making a Clone
object with the call ``\$db->get\_Clone''. In Ensembl a Clone is
really a single sequence entry from EMBL/GenBank. For draft
(unfinished) cases, these are split up into a series of pieces of
contiguous DNA, called ``RawContigs'' in Ensembl terminology. Each
``RawContig'' implements the Bio::SeqI interface of bioperl, meaning
that standard bioperl methods (including dump to flat files) work
fine.


It is a pretty simple case to do a pretty simple task. 

\section{Ensembl Cookbook}


It is probably easier to explain how to use Ensembl by example than going
through the objects first off. Hence the Ensembl cookbook first, followed by 
a more in depth explanation of the objects.

Nearly all scripts follow the same basic pattern

\begin{itemize}
\item get hold of the root dbadaptor 
\item get hold of the Ensembl objects 
\item manipulate them and/or dump them
\end{itemize} 


\subsection{Dump all Protein translations of genes}

\begin{verbatim}

use Bio::EnsEMBL::DBLoader;
use Bio::SeqIO;



# this will die if if can't load this database
$db = Bio::EnsEMBL::DBLoader->new('Bio::EnsEMBL::DBSQL::DBAdaptor/host=kaka.sanger.ac.uk;user=anonymous;dbname=current');


# build seqio output stream

$seqout = Bio::SeqIO->new( '-format' => fasta, '-fh' => \*STDOUT);

@genes = $db->get_all_Gene_id()

foreach $geneid ( @genes ){
      $gene = $db->get_Gene($geneid);
      foreach $trans ( $gene->each_Transcript ) {
          my $pep = $trans->translate();
          # pep is a SeqI compliant object
          $seqout->write_seq($pep);
      }
}

\end{verbatim}

Genes are collections of Transcripts. Each Transcript has one unique
translation. Inside the schema, two Transcripts can share the same
peptide (translation), but the object model does not nicely handle
this. At the moment, Transcripts are only predicted when they have
different peptides (translations).

Notice that getting out these gene objects does not require any
assembly information - we just get the genes out ``directly'' from the
database, irregardless about how the assembly works.

At the moment this is working through the old-style ``central
DBAdaptor'' scheme. In the future we will get out the GeneAdaptor
object and then get the genes out from there. See section \ref{new_adaptor}
for more information about the new adaptor scheme.


\subsection{Dump a region of a chromosome}

This requires access to assembly information. Ensembl can switch
between different assemblies from the same set of underlying DNA
fragments - however in general Ensembl is only distributed with
one assembly. With the object layer one has still indicate which
assembly ``type'' is used - generally this is ``UCSC'' as we use
the assemblies from UCSC.


\begin{verbatim}

use Bio::EnsEMBL::DBLoader;
use Bio::SeqIO;



$db = Bio::EnsEMBL::DBLoader->new('Bio::EnsEMBL::DBSQL::DBAdaptor/host=kaka.sanger.ac.uk;user=anonymous;dbname=current');

# set the assembly type. Ensembl can store multiple assemblies
# concurrently, although in general the gene prediction data will
# only be in sync with one of the assemblies

$db->static_golden_path_type('UCSC');

# get out the staticgoldenpathadaptor. This might be better called
# the virtual contig adaptor

$sadp = $db->get_StaticGoldenPathAdaptor();

# A virtualcontig is a slice of the assembly.

my $vc = $sadp->fetch_VirtualContig_chr_start_end('chr12',10000,20000);

my $seqout = Bio::SeqIO->new( -format => 'fasta',-fh => \*STDOUT);

# write out the virtual contig. $vc is a Bio::SeqI compliant object
# so we can write out it out to a SeqIO stream

$seqout->write_seq($vc);

# alternatively we can get the sequence as a string

$seqstring = $vc->seq();

\end{verbatim}

This script introduces the idea of ``VirtualContigs''. A virtual contig is a
slice of an assembly. A virtual contig can be anything from one tiny piece
of an assembly through to an entire chromosome. (Virtual Contigs are also used
for weirder dumping, including dumping regions via ``clone'' coordinates).

Whatever route to get a virtual contig, one always get the same sort
of object.  The Virtual Contig is-a sequence, with coordinates
starting at one. All the Sequence features (but for genes) you get out
on it are converted to the coordinates of the virtual contig, for example,
starting from one. 

As Virtual Contigs inheriet from Bio::SeqI anything which you can do
with ``standard'' bioperl objects you can do with a virtual contig. In
general you'll wont want to do so much with standard bioperl calls
(such as \$seq->top\_SeqFeatures), but use the more specific Ensembl
calls (such as \$vc->get\_all\_SimilarityFeatures).

One thing which is very different to get are Genes, as Gene does
not inheriet from SeqFeatureI.


\subsection{Dump 50bp around exons of a gene}

This is a more complex operation as the 50bp is now dependent on the
assembly of the genome - a particular exon could be right near a
``switch point'' where the assembly changes from one fragment to the
next (see section \ref{assembly_info} for more information about
assemblies). We have to get a gene out in the context of an assembly.



\begin{verbatim}


$db = Bio::EnsEMBL::DBLoader->new('Bio::EnsEMBL::DBSQL::DBAdaptor/host=kaka.sanger.ac.uk;user=anonymous;dbname=current');


# we expect a gene id as the argument to this script
my $geneid = shift;


# set the assembly type. Ensembl can store multiple assemblies
# concurrently, although in general the gene prediction data will
# only be in sync with one of the assemblies

$db->static_golden_path_type('UCSC');

# get out the staticgoldenpathadaptor. This might be better called
# the virtual contig adaptor

$sadp = $db->get_StaticGoldenPathAdaptor();

# get a VirtualContig around this gene. The 100 is how much
# upstream and downstream from the first and last exons we
# should take

$vc = $sadp->fetch_VirtualContig_of_gene($geneid,100);

# frustratingly here we have now a slice of the genome with our gene
# of interest on it - however there may well be other genes in this
# region, including nested genes inside introns, etc etc.

# need to find ``our'' gene

my $found_gene;

foreach $gene ( $vc->get_all_Genes() ) {
	if( $gene->id eq $geneid ) {
            $found_gene = $gene;
        }
}

if( !defined $found_gene ) {
   die('no gene of interest... on vc. Bad error');
}

# when we get genes on a virtualcontig, the coordinates of the exon are
# magically converted to be relevant to this virtual contig.

foreach $trans ( $found_gene->each_Transcript ) {
 foreach $exon ( $trans->each_Exon ) {

  # there is the theoretical chance that an exon
  # does not lie on this virtualcontig. This would happen
  # when the gene was built on a different assembly from
  # the asembly used to extract them.

  # if the exon is not on the virtual contig, the seqname
  # will not be the same as the virtual contig id

  if( $exon->seqname ne $vc->id ) {
    next;
  }

  if( $exon->strand == 1 ) {
    print " exon ",$exon->id,"  ",$vc->subseq($exon->start-50,$exon->end+50),"\n";
  } else {
    # if we wanted just the sequence of the exon, 
    # $exon->seq would give it to us fine. However we want 50bp each
    # way around

    # no easy way to dump reverse complement. trunc gives us back a Bio::PrimarySeqI
    # object of this region, revcom reverse complements it and seq is the actual sequence
    # this is a little convoluted. 
    print " exon ",$exon->id,"  ",
          $vc->trunc($exon->start-50,$exon->end+50)->revcom->seq,
          "\n";
  }
 }
}

\end{verbatim}

\subsection{Dump rich tab delimited format for one chromosome}

\begin{verbatim}


$db = Bio::EnsEMBL::DBLoader->new('Bio::EnsEMBL::DBSQL::DBAdaptor/host=kaka.sanger.ac.uk;user=anonymous;dbname=current');


# set the assembly type. Ensembl can store multiple assemblies
# concurrently, although in general the gene prediction data will
# only be in sync with one of the assemblies

$db->static_golden_path_type('UCSC');

# get out the staticgoldenpathadaptor. This might be better called
# the virtual contig adaptor

$sadp = $db->get_StaticGoldenPathAdaptor();

# get a virtual contig of the chromosome. This call will take
# a while (30 seconds or so) to complete

$vc = $sadp->fetch_VirtualContig_by_chr_name('chr10');

# dump start,end,strand,type,score,external-identifier

# dump similarity features - mainly BLAST results

foreach $sim ( $vc->get_all_SimilarityFeatures ) {
  print STDOUT join('\t',$sim->start,$sim->end,$sim->strand,$sim->score,$sim->hseqname),"\n";
}

# dump repeats

foreach $rep ( $vc->get_all_RepeatFeatures ) {
  print STDOUT join('\t',$rep->start,$rep->end,$rep->strand,$rep->score,$rep->hseqname),"\n";
}

# dump genes. We'll dump as exons with Transcript id's as the external-identifier

foreach $gene ( $vc->get_all_Genes ) {
  foreach $trans ( $gene->each_Transcript ) {
     foreach $exon ( $trans->each_Exon ) {
       print STDOUT join('\t',$exon->start,$exon->end,$exon->strand,$exon->score,$trans->id),"\n";
     }
  }
}

\end{verbatim}

\section{Ensembl Object Bestiary}

Many Ensembl objects inheriet from Bioperl Interface definitions; they
therefore comply to those interfaces and then extend them in Ensembl
specific ways. The implementation of the objects in general is
\emph{radically} different - most bioperl objects for example just
store everything in a Perl hash, whereas many Ensembl objects are a
Perl hash with minimal information (mainly the primary key of object
in the database) and also access to the underlying database. In these
cases, methods trigger a quick (or sometimes not so quick - if you are
asking for alot of data) trip to the database to retrieve the objects
of interest.

This section has been grouped by the type of Bioperl object the Ensembl
object extends and gives you a quick overview of how to use the methods

\subsection{Bio::SeqI compliant objects}

Bio::SeqI is the bioperl ``heavy'' sequence object, containing the
sequence, identifiers and seqfeatures. The main functions you want to
call from Bio::SeqI compliant objects are the following

\begin{description}
\item[subseq(start,end)] gives back part of a sequence as a string between start and end. The main
frustration with this method is it does not handle the reverse complement - ie you can't put end < start or 
anything to indicate that the reverse strand is needed (this is a Bioperl aspect of the system). You have
to use trunc followed by revcom.
\item[seq] gives back the entire sequence as a string - please be careful only to use this when
you need, as of course it is very easy to have truely massive sequences coming out from the
Ensembl object layer
\item[trunc(start,end)] gives back part of a sequence as a PrimarySeqI object, on which you
can then call other standard methods - in particular, revcom for the reverse complement
\item[top\_SeqFeatures] gives back all the Sequence features. Although this is a very common
thing to do in bioperl, inside Ensembl, this triggers a massive amount of database gets, and you
probably want to call Ensembl specific functions which have a more limited scope.
\end{description}

The other crucial thing you can do with Bio::SeqI compliant objects is write them out
via the SeqIO system. The calls go

\begin{verbatim}
   $seqio = Bio::SeqIO->new( -format => 'embl',-file => '>/an/output/file');
   $seqio->write_seq($BioSeqI_Compliant_Object);
\end{verbatim}

\subsubsection{VirtualContigs}

One of the key Ensembl objects which inheriet from Bio::SeqI is a
Virtual Contig, generally the class
Bio::EnsEMBL::Virtual::StaticContig (the Static means that the golden
path, ie, assembly information is stored in a semi denormalised manner
in the database). If you are looking at for the code for StaticContig
is split between itself and its direct base class,
Bio::EnsEMBL::Virtual::Contig.

Virtual Contigs represented a slice of a chromosome - anything from 
100 Kb through to a entire chromosome. A good example of the use of
a virtual contig is the main ``sequence'' web display, contigview. The web 
script first makes a virtual contig of the region of the chromosome.


The virtual contig presents to the user a Bio::SeqI implementing
object where the sequence starts from 1 and where all the features got
from the virtual contig start are transformed to fit this slice - for
example, if you make a virtual contig from chr2, base pair start
30,000 to 40,000 the first feature, say starting from 30,005 in
chromosome coordinates will come out as start position 5. In general
using this relative coordinate system (ie, virtual contigs always
start at 1) makes coding with virtual contig easier. If you want to go
to global coordinates then you'll have to remember to add the start
position of the virtual contig (in extremis, see if the function
\$vc->\_global\_start is set if you have a virtual contig without
knowing where on the genome it is. This is definitely cheating - some
virtual contigs will \emph{not} have the \_global\_start position set)


One makes a virtual contig from the StaticGoldenPathAdaptor, which should
have really been called VirtualContigAdaptor. The StaticGoldenPathAdaptor 
provides factory functions to make virtual contigs. The only gotcha here is
that you have to set the ``type'' of golden path (assembly) you are using
first - in general Ensembl is only distributed with one assembly and parts
of the data will only make sense with one assembly. The golden path type
has to be set on the root ``DBAdaptor'' globally for your database session.

A snippet of code for making a virtual contig is as follows:

\begin{verbatim}
$db = Bio::EnsEMBL::DBLoader->new('Bio::EnsEMBL::DBSQL::DBAdaptor/host=localhost;user=read_only_user;dbname=test_ensembl');

$db->static_golden_path_type('UCSC');

# get out the staticgoldenpathadaptor. This might be better called
# the virtual contig adaptor

$sadp = $db->get_StaticGoldenPathAdaptor();

$vc = $sadp->fetch_VirtualContig_by_chr_start_end('chr2',30000,40000);
\end{verbatim}

Of course this access point assumes you know where you want to look
first off. There are a couple of other useful entry points, in
parituclar fetch\_VirtualContig\_of\_gene and
fetch\_VirtualContig\_of\_clone (notice that the clone has to be
``golden'', ie that part of the clone is used in the golden
path. Completely redundant clones are not accessible via this call)


Once you have got the VirtualContig, all the standard Bio::SeqI calls are available.
Just to list them here:

\begin{description}
\item[id] provides an auto-generated id for this VirtualContig unique
for the session. This is not meaningful but does comply to Bio::SeqI interface
\item[subseq(start,end)] provides the sequence from start to end. This is executed
efficiently so that if you ask for a small region (say 400bp in a 4MB virtual contig)
only those precise regions are retrieved.
\item[seq] provides the entire sequence. Please try not to use this - subseq is much more
efficient (in fact seq is just a call of subseq(1,length))
\item[top\_SeqFeatures] provides all sequence features - this is rarely used - instead the
methods listed below are more likely to be used.
\end{description}

VirtualContigs also have a number of Ensembl specific calls, which are
shared also by the RawContigs (section \ref{RawContig}). These are
more useful ways to retrieve sequence features. All Sequence features
comply to the Bio::SeqFeatureI interface, so the following methods are
always available on them

\begin{description}
\item[start] start position of a sequence feature
\item[end] end position (inclusive coordinates) of a sequence
feature. ie, a sequence feature 1,2 is the first two bases of a
sequence. (this coordinate system is the ``standard'' bioinformatics coordinate system
and although it would be nicer to have a C-style coordinate system, far too much existing data
has been written in this coordinate system for us to buck the trend)
\item[strand] 1,-1 or 0 - 0 means strand is meaningless, eg in a low complexity repeat
\item[length] self explanatory
\item[seq] returns a Bio::PrimarySeqI object of this SeqFeature's sequence, automatically reverse complemented if
needed. \$seqfeature->seq->seq is therefore the sequence as string of a SeqFeature
\item[entire\_seq] returns a Bio::PrimarySeqI object of this SeqFeature's underlying sequence, ie where the
seq string starts at 1
\end{description}
 
When sequence features are retrieved from a VirtualContig the sequence features are presented
in the VirtualContig coordinate system. When a feature is half-on/half-off a VirtualContig it
is simply not reported (which might be a mistake)

\begin{description}

\item[get\_all\_SimilarityFeatures] provides all BLAST hits and other
database hits, including other features which we can't
classify. SimilarityFeatures are FeaturePairs (See section
\ref{FeaturePairs}).

\item[get\_all\_RepeatFeatures] provides all repeats - generally the
output of repeatmasker. RepeatFeatures are also FeaturePairs (See
section \ref{FeaturePairs}).

\item[get\_all\_PredictionFeatures] provides all a series of ab-initio
features - general GenScan results, which is given back a SeqFeature
with a list of sub\_SeqFeatures

\end{description}

It is likely that in the next generation of the object layer these
series of ``canned queries'' will end up being a more dynamic query
interface. Stay tuned on ensembl-dev if you are interested.


\subsubsection{RawContigs}

Genomes are made from underlying clone-based sequences which are
deterimined in a number of ways. In the case of the human genome, one
natural data representation are the fragments made from Phrap (or
another clone-orientated assembler) which are then deposited inside
EMBL/GenBank. Inside Ensembl each of these fragments is called a
``RawContig''. For a rough draft clone there will be multiple
RawContigs per clone. For a finished clone there will be one
RawContig per clone.

Ensembl has an almost identical interface to RawContigs as VirtualContigs:
all the methods listed above except for the \_global\_start methods are valid.
They are listed to be complete below:


\begin{description}

\item[id] provides the Ensembl stringified id for the RawContig,
currently Accession.uniquenumber but about to change to something
like accession.verison.start.end


\item[subseq(start,end)] provides the sequence from start to end. Like in VirtualContigs,
this is executed efficiently, so that only this region is retrieved from the database.

\item[seq] provides the entire sequence. Please try not to use this - subseq is much more
efficient (in fact seq is just a call of subseq(1,length))

\item[top\_SeqFeatures] provides all sequence features - this is rarely used - instead the
methods listed below are more likely to be used.

\item[get\_all\_SimilarityFeatures] provides all BLAST hits and other
database hits, including other features which we can't
classify. SimilarityFeatures are FeaturePairs (See section
\ref{FeaturePairs}).

\item[get\_all\_RepeatFeatures] provides all repeats - generally the
output of repeatmasker. RepeatFeatures are also FeaturePairs (See
section \ref{FeaturePairs}).

\item[get\_all\_PredictionFeatures] provides all a series of ab-initio
features - general GenScan results, which is given back a SeqFeature
with a list of sub\_SeqFeatures

\end{description}





\end{document}







